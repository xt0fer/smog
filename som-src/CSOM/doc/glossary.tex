\section{Glossary}
\label{sec:glossary}


\begin{description}
\item[bytecode] A bytecode
\item[bytecode index] short: bc\_idx
\item[class] The word ``class'' can refer to two distinct
  interpretations:
  \begin{enumerate}
  \item OO class \emph{or \C class}
  \item \SOM class
  \end{enumerate}
\item[\C class] The combination of a  \C  \lstinline|struct| defining
  data fields, a VTABLE defintion, prototypes of ``class methods''  and the corresponding
  implementations.

  A class shall be refered to by its component name and its class name;
  thus \class{vmobjects}{VMObject} would refer to the files
  \file{src/vmobjects/VMObject.h} (containing the \lstinline|struct|
  defintions and class method prototypes) and
  \file{src/vmobject/VMObject.c} (containing the methods' implentations
  and the VTABLE).
\item[\CSOM] This can refer
  \begin{enumerate}
  \item to the entire project ``\CSOM'',
  \item to the project's entire code base,
  \item or to the compiled VM binary file ``\CSOM'', which can be
    located either in the \dir{src} or the \dir{build} directory.
  \end{enumerate}
\item[frame]~
\item[object]~
\item[OO]~
\item[Object Model]~
\item[OO Object] These terms refer to the ``Object Model'' used
  throughout \CSOM.
\item[Primitive] A ``primitive'' may refer to different meanings:
  \begin{enumerate}
  \item The \lstinline[language=SOM]|primitive|-Keyword of the \SOM language
  \item A \SOM method, which is solely defined though the keyword mentioned
  \item A single \C function located in an external binary file which is linked
    to a \SOM method on or after compiling.
  \item A colletion of such \C funtions in a binary dynamically loadable library.
  \end{enumerate}
\item[\SOM class] A ``\SOM class'' is a class defined in the \SOM language,
  written to a file. When loaded, it is represented by an instance of
\class{vmobjects}{VMClass}.
\item[string]~
\item[String] An instance of \class{misc}{String}. This is an OO-capable
  wrapper around C-strings. They shall be used in favor of C-strings.
\item[C-string] a `$\backslash 0$'-terminated sequence of bytes. Don't
  rely on signedness of them. Their direct use is discouraged. Use 
  (OO-)Strings instead.
\item[Trait] 
A trait consits of a VTABLE and an empty (except the VTABLE
pointer) Object definition with corresponding method implementations.

To assign a trait to classes, use\\
\lstinline|ASSIGN_TRAIT(aTrait, aClass)| \\
in their VTABLE-functions.


Use \lstinline|TSEND(aTrait, anObject, aMessage, ...)| to send an
trait-message to an object. \\
Do not use \lstinline|SEND|.

Use \lstinline|SUPPORTS(anObject, aTrait)| to check, whether an object
supports an specific interfaces.

Currenty, a class can only support one trait at a time. But each
trait would be able to support another trait via the same mechanism of
\lstinline|ASSIGN_TRAIT|.

See \class{vmobjects}{VMInvokable} for a trait implementation example and
\class{vmobjects}{VMPrimitive} and \class{vmobjects}{VMMethod} for examples of
trait usage. \\
The macros mentioned are defined in \file{src/vmobjects/OOObject.h}.

This whole concept may change in the future.
\item[vtable] see virtual method table
\item[VTABLE] see virtual method table
\item[virtual method table] A structure unique to each individual
  class containing \C function pointers.
\end{description}
