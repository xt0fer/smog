%
% the CSOM Style Guide
%
% $Id: styleguide.tex 169 2008-01-03 15:07:48Z tobias.pape $
%

\section{\CSOM Style Guidelines}
\label{sec:csom-style-guid}

\subsection{Common Rules}
\label{sec:common-rules}
\begin{itemize}
\item A line shall have at most 80 columns.
\item Each file shall start with copyright and licensing information.
  Just before both, the string \texttt{\$Id\$} shall be placed as a
  comment.
\item The sole language of coding and documentation is
  \emph{English}.
\ignore{
\begin{lstlisting}
// do this
void copy(void *source, void *dest) {
    /*
     * copying from *source space to *dest space
     *
     */
     ...
}

// not this
void kopieren(void *origine, void *destinacion) {
    /*
     * effingo ex *origine ad *destinacion
     *
     */
     ...
}
\end{lstlisting}
}
\item A file shall end with a new line.
\end{itemize}


\subsection{Rules for \C}
\label{sec:rules-c}

The following rules apply to all \C sourcecode files.

\begin{itemize}
\item The language dialect used is ISO C99  with GNU extensions.
\item \emph{Avoid code duplication.}
\item All \lstinline|#endif|-statement shall be appended with a
  name. For mactros encpasulation a header file, this is the filename
  formatted as \lstinline|FILENAME_H_|.
\lstinline|#ifdef ... #endif| blocks shall start a new indentation
level.
\begin{lstlisting}
// do this
#ifdef DEBUG
    i = 10;
#endif DEBUG

// not this
#ifdef DEBUG
    i = 10;
#endif /*DEBUG*/

// nor this
#ifdef DEBUG
i = 10;
#endif
\end{lstlisting}
\item There shall be exactly one space between a function's type and
  its name.
\begin{lstlisting}
// do this
void print(char *fmt, ...) {
    ...
}

// not this
void    print(char *fmt, ...) {
    ...
}
\end{lstlisting}
\item When definig functions, in a functions signature, the opening
  brace \lstinline|{| shall be on the  same line as the corresponding
    closing paren \lstinline|)|. There shall be exactly one space between them.
\begin{lstlisting}
// do this
void print(char *fmt, ...) {
    ...
}

// not this
void print(char *fmt, ...){
    ...
}

// nor this
void print(char *fmt, ...)
{
    ...
}

// and not this
void print(char *fmt, ...)
        {
    ...
        }
\end{lstlisting}
\item An opening paren \lstinline|(| shall not be appended, a closing
  paren \lstinline|)| shall not be prepended with a space.
\begin{lstlisting}
// do this
bool right = ((a || b) && (c ^^ d)) || (e > f);

// not this
bool right = ( ( a || b ) && ( c ^^ d ) ) || ( e > f ) ;
\end{lstlisting}
\item Functions shall not start with empty lines. Code or comments
  shall start immediately.
\begin{lstlisting}
// do this
void print(char *fmt, ...) {
    int i=0;
    ...
}

// not this
void print(char *fmt, ...) {

    int i=0;
    ...
}
\end{lstlisting}
\item Each function definition is preceded and followed by two blank lines.
\begin{lstlisting}
// do this


void foo1(...) {
    ...
}


void foo2(...) {
    ...
}


//not this
void foo1(...) {
    ...
}
void foo2(...) {
    ...
}
\end{lstlisting}
\item \lstinline|**| and \lstinline|*| shall not be unnecessarily
  surrounded by spaces.
\begin{lstlisting}
// do this
array = *another;

// not this
array = * another;
\end{lstlisting}
\item When casting, the closing paren \lstinline|)| after the target type shall not
  be appended by a space.
\begin{lstlisting}
// do this
foo = (char*)bar;

// not this
foo = (char*) bar;
\end{lstlisting}
\item Redundant parens shall be avoided.
\begin{lstlisting}
// do this
return 0 == (a - b);

// not this
return (0 == (a - b));
\end{lstlisting}
\item In pointer definitions, the asterisk \lstinline|*| shall be
  aligned to the type not to the variable's name.
\begin{lstlisting}
// do this
int* foo;

// not this
int *foo;

// nor this
int * foo;
\end{lstlisting}
\item Comments like \lstinline|//end of somefile.c| shall be avoided.
\item A space shall be placed behind commas \lstinline|,| in parameter
  lists et al.
\begin{lstlisting}
// do this
foo(a, b, c);

// not this
foo(a,b,c);

// nor this
foo(a ,b ,c);
foo(a , b , c);
\end{lstlisting}
\item Operators, except \lstinline|!|, shall be surrounded by spaces.
\begin{lstlisting}
// do this
bool right = !((a || b) && (c ^^ d));

// not this
bool right = ! ((a||b)&&(c^^d));
\end{lstlisting}
\item \lstinline|#pragma mark| code separators are preceded and followed by two blank lines.
\begin{lstlisting}
//do this
void foo(...) {
    ...
}


#pragma mark Helper functions


int helper(...) {
    ...
}

// not this
void foo(...) {
    ...
}
#pragma mark Helper functions
int helper(...) {
    ...
}
\end{lstlisting}
\item Backslashes \lstinline|\| shall be prepended by a space if
  encountered on line endings.
  \begin{lstlisting}
// do this
#define MY_LONG_MACRO_FORMAT \
    int i; \
    int j; \
    int k

// not this
#define MY_LONG_MACRO_FORMAT\
    int i;\
    int j;\
    int k
\end{lstlisting}
\item There shall not be a space between \lstinline|if/for/while/anyFunctionName|
  and the following opening paren \lstinline|(|.
  \begin{lstlisting}
// do this
if(a == b) {
    while(x > 2) {
        z = foo(x);
    }
}

// not this
if (a == b) {
    while (x > 2) {
        z = foo (x);
    }
}
\end{lstlisting}
\item \lstinline|#define| blocks shall not end with semi colon
  \lstinline|;| if the macro defined will be used with semi colons
  later on. This apllies especially to the \lstinline|_FORMAT|
  macros.
  \begin{lstlisting}
// do this
#define MY_LONG_MACRO_FORMAT \
    int i; \
    int j; \
    int k

// not this
#define MY_LONG_MACRO_FORMAT \
    int i; \
    int j; \
    int k;
\end{lstlisting}
\item Function comments shall be written in \verb|javadoc|-style,
  i.~e. at least
  \begin{lstlisting}
/**
 * brief description of what the function does
 */
\end{lstlisting}They are \emph{immediately} followed by the function header without any blank lines.
\item Each block of line comments shall start and end with a empty comment
  line. Such blocks are also preceded and followed by two blank lines.
\begin{lstlisting}
// do this


//
// some useful commentary
//


// not this
// some less useful commentary
\end{lstlisting}
\item There are no TABs (i.~e. no ASCII character 09). The indentation
  width shall be 4 spaces.
\begin{lstlisting}[showspaces]
// do this
void foo(void) {
    int i;
    if(a==...) {
        while(true) {
            ...
        }
        ...
    }
}
\end{lstlisting}
\lstinputlisting[showtabs]{tab.c}
\item Lines which are too long to fit the 80 columns shall be split, such
  that it remains sensefully grouped. For assignation, this means to try
  to keep the right side of it together.
  \begin{lstlisting}
// do this
int foo =
    very_long_function_name(first_parameter, second_parameter, third_parameter);

// not this
// do this
int foo =  very_long_function_name(first_parameter, second_parameter,
    third_parameter);
  \end{lstlisting}
\item When using includes from different \CSOM-directories, they shall
  be grouped and parted by empty lines.
  \begin{lstlisting}
// do this
#include <memory/gc.h>

#include <misc/String.h>

#include <vmobjects/VMFrame.h>
#include <vmobjects/VMBigInteger.h>
#include <vmobjects/VMMethod.h>

#include <compiler/Parser.h>

// not this
#include <compiler/Parser.h>
#include <misc/String.h>
#include <vmobjects/VMBigInteger.h>
#include <vmobjects/VMMethod.h>
#include <memory/gc.h>
#include <vmobjects/VMFrame.h>
\end{lstlisting}
\end{itemize}

\subsection{Rules fom \SOM}
\label{sec:rules-fom-som}

\subsection{Rules for Make et al.}
\label{sec:rules-make-et}

\begin{itemize}
\item \emph{Do not use recursive makes.}
\end{itemize}